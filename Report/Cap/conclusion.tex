In the report,  the Breadth-First Search algorithm has been analyzed and some possible parallel solutions to improve the performance of the sequential implementation have been proposed.
The problem analysis leads to determine that the frontier visit is the best spot for the parallelization. To achieve it, a farm-based approach has been used, in particular, three solutions were developed: static and dynamic scheduling using pthread and static scheduling with Fast Flow version.

The evaluation of the results shows that with a static partitioning, that splits the frontier based on the number of workers in the farm, presents a worse load balancing. This scenario occurs especially with a low number of workers and with very dense graphs. 
To solve this problem, a dynamic scheduling version has been proposed that allows to divide the level into smaller chunks and using a shared data structure between workers to manage the workload in a more balanced way. 
In Fast Flow the obtained results are very similar to the static pthread version, as they share the same workload management, with a drop in performance for numerous farms that carry a greater overhead due to the communication mechanisms.
The behavior of the solutions on the tested graphs leads to believe that with larger number of nodes it could be possible, in principle, to obtain remarkable speedups, therefore they could be preferred to a sequential version.